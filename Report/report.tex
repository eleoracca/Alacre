\documentclass[11pt,a4paper]{article}
\usepackage{RaccaStyle}
\usepackage{RSProgrammazione}

\begin{document}
\thispagestyle{empty}

\rmfamily
\begin{center}
\ \\
\vspace{2cm}

\LARGE{\textcolor[rgb]{1,0,0}{RELAZIONE sul PROGETTO\\ per l'ESAME di\\TECNICHE di ANALISI NUMERICA e SIMULAZIONE}\\}
\huge{\textcolor[rgb]{1,0,0}{\ \\Simulazione di un rivelatore di vertice}\\}
\hrulefill \\
\vspace{1.5cm}

\Large{Anno Accademico 2018/2019
\\ \ \\
Professore: MASERA MASSIMO
\\ \ \\ 
Candidate: RACCA ELEONORA\\
\ \ \ \ \ \ \ \ \ \ \ \ \ SAUDA CRISTINA}

\vspace{6cm}
\begin{figure}[h]
\centering
	\includegraphics[width=0.35\textwidth]{logounito.pdf}
\end{figure}

\end{center}

\section{Introduzione}
\par Qualcosa di generico sull'obiettivo del progetto

\section{Struttura del programma}
\par Qualcosa sulle classi
\begin{itemize}
\item \lstinline{Punto}
\item \lstinline{Rivelatore}
\item \lstinline{Trasporto}
\item \lstinline{Urto}
\item \lstinline{Vertice}
\end{itemize}

\section{Efficienza di ricostruzione}

\end{document}

%-------- RIFERIMENTI --------
%\figurename~
%\tablename~

%-------- FIGURE --------
%\begin{wrapfigure}{r}{0.4\textwidth}
%\vspace{-60pt}
%\centering
%\includegraphics[width=0.25\textwidth]{Immagini/Inverter.pdf}
%\vspace{-15pt}
%\caption{Schema della piedinatura dell'integrato NOT SN74LS04}\label{SN7404}
%\vspace{5pt}
%\includegraphics[width=0.25\textwidth]{Immagini/SN7400.pdf}
%\vspace{-15pt}
%\caption{Schema della piedinatura dell'integrato NAND SN7400}\label{SN7400}
%\vspace{-5cm}
%\end{wrapfigure}
%
%\begin{figure}[!h]
%\centering
%\includegraphics[width=0.60\textwidth]{Immagini/cisVs.pdf}
%\caption{Andamento di $V_{out}$ in funzione della variazione di $V_B$ per una porta NAND}
%\label{Grafico}
%\end{figure}

%-------- TABELLE --------
%\begin{table}[h!]\centering
%\caption{Tabella della presa dati per lo studio della transizione tra gli stati high e low}\label{transizione alto basso}
%\begin{tabular}{|c|c|c|c|}
%\hline
%\textbf{Tensione in ingresso} & \textbf{Errore} & \textbf{Tensione in uscita} & \textbf{Errore}\\
%$V_B$ & $\sigma_{V_B}$ & $V_{out}$ & $\sigma_{V_{out}}$\\
%$V$ & $V$ & $V$ & $V$\\
%\hline
%0,231 & 0,001 & 3,68 	& 0,01\\
%0,536 & 0,001 & 3,68 	& 0,01\\
%0,760 & 0,001 & 3,66 	& 0,01\\
%0,862 & 0,001 & 3,66	& 0,01\\
%0,825 & 0,001 & 3,64 	& 0,01\\
%\hline
%\end{tabular}
%\end{table}